\documentclass[conference]{IEEEtran}
\IEEEoverridecommandlockouts
% The preceding line is only needed to identify funding in the first footnote. If that is unneeded, please comment it out.
\usepackage{listings}
\usepackage{color}

\definecolor{dkgreen}{rgb}{0,0.6,0}
\definecolor{gray}{rgb}{0.5,0.5,0.5}
\definecolor{mauve}{rgb}{0.58,0,0.82}

\lstset{frame=tb,
	language=C++,
	aboveskip=3mm,
	belowskip=3mm,
	showstringspaces=false,
	columns=flexible,
	basicstyle={\small\ttfamily},
	numbers=none,
	numberstyle=\tiny\color{gray},
	keywordstyle=\color{blue},
	commentstyle=\color{dkgreen},
	stringstyle=\color{mauve},
	breaklines=true,
	breakatwhitespace=true,
	tabsize=3
}
\usepackage{cite}
\usepackage{amsmath,amssymb,amsfonts}
\usepackage{algorithmic}
\usepackage{graphicx}
\usepackage{textcomp}
\usepackage{xcolor}
\def\BibTeX{{\rm B\kern-.05em{\sc i\kern-.025em b}\kern-.08em
    T\kern-.1667em\lower.7ex\hbox{E}\kern-.125emX}}
\begin{document}

\title{\'Arbol ancestral \\
{\footnotesize \textsuperscript{}}
\thanks{}
}

\author{\IEEEauthorblockN{1\textsuperscript{st}Pe\~na Andia, Victor Angelo}
\IEEEauthorblockA{\textit{Ciencias de la computaci\'on} \\
\textit{UTEC}\\
Lima, Peru \\
victor.pena@utec.edu.pe}
\and
\IEEEauthorblockN{2\textsuperscript{nd} Villanueva Avellanada, Gilmar}
\IEEEauthorblockA{\textit{Ciencias de la computaci\'on} \\
\textit{UTEC}\\
Lima, Peru \\
gilmar.villanueva@utec.edu.pe}
\and
\IEEEauthorblockN{3\textsuperscript{rd} Tanta Villanueva, Johan Hugo}
\IEEEauthorblockA{\textit{Ciencias de la computaci\'on} \\
\textit{UTEC}\\
Lima, Peru \\
johan.tanta@utec.edu.pe}
}

\maketitle

\begin{abstract}
En este proyecto que se implemento en c++ un traductor de t\'erminos ancestrales. Para esto se creo una gram\'atica capaz de detectar cada una de las palabras y poder luego traducirlas. 

\end{abstract}

\begin{IEEEkeywords}
gram\'atica, C++, traductor, t\'erminos ancestrales 
\end{IEEEkeywords}

\section{Introduction}
Los lenguages son un conjunto de s\'imbolos definidos por $\sum$, 
donde con los elementos en sigma es posible crear cadenas de s\'imbolos finitas. Pero, para conocer si una determinada oraci\'on o conjunto de palabras est\'an formadas correctamente se usa el concepto de gram\'atica. Este concepto ayuda a definir correctas cadenas que puedan ser reconocidas por un aut\'omata. 

Estos conceptos ayudan a definir un determinada maquina capaz de resolver uno o unos ciertos problemas, dependiendo de sus estados.

Para este problema se uso los conceptos anteriores para si conseguir que una cadena de palabras sea reconocida. Luego, esta cadenas sea inspeccionada si es correcta. Finalmente, si es correcta se muestra la traducci\'on de la frase, caso contrario se presenta un error y se interrumpe el proceso.

Por otro lado, se creo una representaci\'on intermedia que expresa la relaci\'on de padre o madre respecto a un funci\'on.

\section{Planteamiento del problema}

Para resolver el problema existe un conjunto infinito A1 que tiene los elementos ancestrales a traducir $A1 = \{mother, father, grandmother, greatgrandfather, ...\}$, 
dentro de este conjunto se identifico las palabras con las que la gram\'atica trabaja: father, mother, great y grand.

Con estos datos se analizo la formaci\'on de una traducci\'on intermedia de tipo:
$mother = mo() o father = fa()$ o,
$grandmother = g(mo())$. 

Al tener las palabras pertenecientes a la gram\'atica y la conversi\'on de palabra a representaci\'on intermedia, es posible diseñar la gram\'atica adecuada para este lenguaje. La gram\'atica diseñada seria de este modo:

$$S-> G M | G F  $$
$$G-> R D $$
$$R-> R A $$
$$F-> father$$
$$M-> mother$$
$$D-> grand$$
$$A-> great$$

Al tener las reglas por las cuales la gram\'atica se forma es posible comenzar de derecha a izquierda a analizar la palabra.

En la segunda parte del problema se pide analizar la respuesta de la primera parte y traducirlo. Para esto la funci\'on debe hacer uso de una gram\'atica similar a la usada para los imput's

$$S-> G M | G V  $$
$$G-> R D $$
$$R-> R A $$
$$V-> vatter$$
$$M-> mutter$$
$$D-> gross$$
$$A-> ur$$

Con la gram\'atica anterior y el resultado de la funci\'on se logro traducir la frase y encontrar una representaci\'on intermedia.

Finalmente, se recibe una oraci\'on cuyas palabras est\'an dentro del conjunto A3, siendo $A3 = \{The, mother, father, of, Mary, John \}$. Esta oraci\'on pasara traducirse al lenguaje A4, donde $A4 = \{Die, ein, mutter, vatter, von, Maria, John \}$

La gram\'atica del lenguaje A3 es:

$$Sentence    -> Init Sentence2$$
$$Sentence2   -> Second Sentence2 | O Nombre $$
$$Nombre      -> Mary | John$$
$$Init        -> The mother | The father $$
$$Second      -> of the mother | of the father$$ 
$$O           -> of$$

La gram\'atica del lenguaje A4 es:
$$Sentence2   -> Second.Sentence2 | G O Nombre $$
$$Nombre      -> Mary | John$$
$$Second      -> ur $$ 
$$O           -> von$$
$$G -> gross$$

Con estas gram\'aticas creadas solo es necesario realizar un c\'odigo que al leer las frases verifique que sean correctas, mientras se encarga de traducirlas.

\section{Resoluci\'on}

El problema lo dividimos en dos principales partes, donde la primera se usa dos clases y en la segunda dos funciones. Se usa la funci\'on $$string \  middleInterpretation(string \ word)$$
donde recibe como par\'ametro $word$ que es la palabra que pasara a ser una funci\'on. 

\begin{lstlisting}

// functions.cpp
string middleInterpretation(string word){
	bool controlFoM = false, controlG = false;
	string partialWord;
	string middle;
	for (auto letter = word.rbegin(); letter != word.rend() ; ++letter) {
		partialWord.insert(partialWord.begin(),*letter);
		if (partialWord == "father" and !controlFoM){
			middle.insert(0,"fa()");
			partialWord = "";
			controlFoM = true;
		}else if(partialWord == "mother" and !controlFoM){
			middle.insert(0,"mo()");
			partialWord = "";
			controlFoM = true;
		} else if(partialWord == "grand" and !controlG){
			middle.insert(0,"g(");
			middle.insert(middle.end(),')');
			partialWord = "";
			controlG = true;
		}else if(partialWord == "great" and controlG){
			middle.insert(0,"g(");
			middle.insert(middle.end(),')');
			partialWord = "";
		}else{
			if(partialWord.size() > 6 ){
				cout<<"Error en la frase: "<<partialWord<<endl;
				return "error";
			}
		}
	}
	if(!partialWord.empty()){
		cout<<"Error en la frase: "<<partialWord<<endl;
		return "error";
	}
	return middle;
}

\end{lstlisting}
	
En esta funcion, la variable locales controFoM se encarga de controlar si la frase ya recibio "father" o "mother", ya que esto se debe recibir solo una vez. Por otro lado, controlG se encarga de asegurarse que se reciba solo una vez "grand" y los demas sean "great". Luego, partialWord es la palabra que se forma concatenando cada caracter de derecha a izquierda. Los casos posibles en que la palabra parcial sea distinta a lo esperado se retorna "error". Si todo finalizaba como lo esperaba se devolvia la representacion intermedia.

Para la segunda parte del problema se uso una funcion que recibia la representacion intermedia y luego la traducia a aleman. Para esto se uso la funcion: 
$$string\  finalInterpretation(string\  word) $$
. La funcion recibe el string $word$ que es la representacion intermedia a traducir. 

\begin{lstlisting}
	
//functions.cpp
string finalInterpretation(string word){
	int count = 0;
	bool controlFoM = false, controlG = false;
	string partialWord;
	string middle;
	for (auto letter = word.rbegin(); letter != word.rend() ; ++letter) {
		partialWord.insert(partialWord.begin(), *letter);
		if (partialWord == ")") {
			count += 1;
			partialWord = "";
		}else if (partialWord == "fa(" and !controlFoM and count>0){
			middle.insert(0,"vatter");
			partialWord = "";
			count -=1;
			controlFoM = true;
		}else if(partialWord == "mo(" and !controlFoM and count>0){
			middle.insert(0,"mutter");
			partialWord = "";
			controlFoM = true;
			count -=1;
		} else if( partialWord == "g(" and controlFoM and !controlG and count>0){
			middle.insert(0,"gross");
			partialWord = "";
			controlG = true;
			count -=1;
		}else if (partialWord == "g(" and controlFoM and controlG and count>0){
			middle.insert(0,"ur");
			partialWord = "";
			count -=1;
		}else{
			if(partialWord.size() > 3 or count == 0){
			cout<<"Error en la frase: "<<partialWord<<endl;
			return "error";
			}
		}
	}
	return middle;
}

\end{lstlisting}

Las variables boleanas se usarn para controlar que la funcion input sea la adecuada. En caso no cumpla se retorna error. Por otro lado, la variable count sirve para tener en cuenta la cantidad de ")" para controlar la cantidad de funciones recibidas. Finalmente, al igual que la funcion anterior se itera de derecha a izquierda para obtener las funciones necesarias. El resultado sera la traduccion final en caso cumpla todos los requisistos o un mensaje de error.

El codigo ignora primero todos los caracteres ")", solo lleva cuenta de estos para poder identificar el numero de funciones que debe existir. Despues exige que la primera palabra leida sea "mo" o "fa, estos solo los recibe una vez ya que el control cambiara e impedira que pueda ingresarse mas veces. Luego sigue la cantidad de "g" que continua, si es mayor a la esperada se lanzara un error. La ultima condicional se encarga de devolber el error, en la version actualizada es posible mostrar en que parte de la linea ingresada esta el error. 

La parte final del proyecto se uso las mismas tecnicas, la principal diferencia es el uso de una clase adicional. La primera clase, dictionary es una estructura que nos almacena las palabras en ingles. Su unica funcion nos permite obtener si la palabra esta dentro de dictonary. Por otro lado, la clase parse se usa para contener las palabras del otro lenguage y usarlas para contruir la frase traducida.

\begin{lstlisting}
//parse.cpp
string parser::parse(string sentenceToTranslate) {
string translatedSentence;
string partialWord;
bool theControl = false;
bool nameControl = false;
bool ofControl = false;
bool control = false;   //controls the first "of"
bool isFather = false;  //controls if the last word read is father, else is mother
int count = 0;          //count the number of "father" or "mother" read
for (auto letter = sentenceToTranslate.rbegin();letter != sentenceToTranslate.rend(); letter++  ) {
if(*letter != ' '){
partialWord.insert(partialWord.begin(), *letter);
}
if (wordsToTranslate->isInDictionary(partialWord)){
if ((partialWord == "Mary" or partialWord == "John") and !nameControl and !theControl){
translatedSentence.insert(0, partialWord);
partialWord = "";
nameControl = true;
} else if( partialWord == "of" and nameControl and  !ofControl ){
if (!control and !theControl){
translatedSentence.insert(0,translatedWords->at(6)+" ");
partialWord = "";
control = true;
ofControl = true;
theControl = false;
}else {
partialWord = "";
ofControl = true;
theControl = false;
}
} else if((partialWord == "father" or partialWord == "mother") and nameControl and ofControl){
count +=1;
isFather = partialWord == "father";
partialWord = "";
ofControl = false;
} else if (partialWord == "the" and !theControl){
partialWord = "";
theControl = true;
} else if(partialWord == "The"){
string grossurMaker;
if (isFather){
grossurMaker.insert(0,"vater ");
} else{
grossurMaker.insert(0,"mutter ");
}
for (int contador = 0; contador < count-1 ; ++contador) {
if(contador == 0){
grossurMaker.insert(0, "gross");
} else{
grossurMaker.insert(0,"ur");
}
}
translatedSentence.insert(0,grossurMaker);
if(count==1){
translatedSentence.insert(0,"Die ");
} else{
translatedSentence.insert(0,"Ein ");
}
partialWord ="";
}else{
if(partialWord.size() > 7){
cout<<"Error en la frase: "<<partialWord<<endl;
return "error";
}
}
}else{
if(partialWord.size() > 7){
cout<<"Error en la frase: "<<partialWord<<endl;
return "error";
}
}
}
if(!partialWord.empty()){
return "error";
}
return translatedSentence;
}

\end{lstlisting}

En la funcion presentada se uso variables  boolenanas que funcionan cambiando de valor para aceptar la combinacion adecuada de palabras, son la manera de aceptar o rechazar lo que cumpla o incumpla la gramatica.

Para que pueda hacer lo anterior las variables cambian respecto a la palabra reconocida. Primero espera reconocer un nombre para asi poder iniciar con lo demas. Cuando recibe "John" o "Mary" recien es posible que el programa pueda recibir of, activando el siguiente control. El control de la palabra "of" esta activo hasta que recibe "father" or "mother". Tambien cuando recibe lo anterior el control de father y mother es activado por lo que hace posible que el programa siga la gramatica indicada.

\section{Concluciones}

En conclusion, si es posible diseñar la gramatica de un lenguage es posible diseñar un programa capaz de reconocerlo. Ademas, como se demostro en el codigo, es posible generar "estados" mediate el uso de variables. Todo esto gracia a que en un inicio es posible contruir la gramatica y luego seguir esta como si de un automata se tratase. La dificulta empieza cuando la gramatica es mas grande y el lenguage puede generar ambiguedades. Es por esto que en un inicio del problema se hizo la gramatica correspondiente para asi tener una guia para la resolucion del problema.

\end{document}
