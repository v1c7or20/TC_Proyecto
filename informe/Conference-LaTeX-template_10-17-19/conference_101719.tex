\documentclass[conference]{IEEEtran}
\IEEEoverridecommandlockouts
% The preceding line is only needed to identify funding in the first footnote. If that is unneeded, please comment it out.
\usepackage{listings}
\usepackage{color}

\definecolor{dkgreen}{rgb}{0,0.6,0}
\definecolor{gray}{rgb}{0.5,0.5,0.5}
\definecolor{mauve}{rgb}{0.58,0,0.82}

\lstset{frame=tb,
	language=C++,
	aboveskip=3mm,
	belowskip=3mm,
	showstringspaces=false,
	columns=flexible,
	basicstyle={\small\ttfamily},
	numbers=none,
	numberstyle=\tiny\color{gray},
	keywordstyle=\color{blue},
	commentstyle=\color{dkgreen},
	stringstyle=\color{mauve},
	breaklines=true,
	breakatwhitespace=true,
	tabsize=3
}
\usepackage{cite}
\usepackage{amsmath,amssymb,amsfonts}
\usepackage{algorithmic}
\usepackage{graphicx}
\usepackage{textcomp}
\usepackage{xcolor}
\def\BibTeX{{\rm B\kern-.05em{\sc i\kern-.025em b}\kern-.08em
    T\kern-.1667em\lower.7ex\hbox{E}\kern-.125emX}}
\begin{document}

\title{\'Arbol ancestral *\\
{\footnotesize \textsuperscript{*}Note: Sub-titles are not captured in Xplore and
should not be used}
\thanks{Identify applicable funding agency here. If none, delete this.}
}

\author{\IEEEauthorblockN{1\textsuperscript{st}Pe\~na Andia Victor Angelo}
\IEEEauthorblockA{\textit{Ciencia de la computaci\'on} \\
\textit{UTEC}\\
Lima, Peru \\
victor.pena@utec.edu.pe}
\and
\IEEEauthorblockN{2\textsuperscript{nd} Given Name Surname}
\IEEEauthorblockA{\textit{dept. name of organization (of Aff.)} \\
\textit{name of organization (of Aff.)}\\
City, Country \\
email address or ORCID}
\and
\IEEEauthorblockN{3\textsuperscript{rd} Given Name Surname}
\IEEEauthorblockA{\textit{dept. name of organization (of Aff.)} \\
\textit{name of organization (of Aff.)}\\
City, Country \\
email address or ORCID}
}

\maketitle

\begin{abstract}
En este proyecto que se implemento en c++ un traductor de t\'erminos ancestrales. Para esto se creo una gram\'atica capaz de detectar cada una de las palabras y poder luego traducirlas. 

\end{abstract}

\begin{IEEEkeywords}
gram\'atica, C++, traductor, t\'erminos ancestrales 
\end{IEEEkeywords}

\section{Introduction}
Los lenguages son un conjunto de simbolos definidos por $\sum$, 
donde con los elementos en sigma es posible crear cadenas de simbolos finitas. Pero, para conocer si una determinada oracion o conjunto de palabras estan formadas correctamente se usa el concepto de gram\'atica. Este concepto ayuda a definir correctas cadenas que puedan ser reconocidas por un automata \cite{b1}. 

Estos conceptos ayudan a definir un determinada maquina capaz de resolver uno o unos ciertos problemas, dependiendo de sus estados \cite{b2}.

Para este problema se uso los conceptos anteriores para si conseguir que una cadena de palabras sea reconocida. Luego, esta cadenas sea inspeccionada si es correcta. Finalmente, si es correcta se muestra la traduccion de la frase, caso contrario se presenta un error y se interrumpe el proceso.

Por otro lado, se creo una representacion intermedia que expresa la relacion de padre o madre respecto a un funcion.

\section{Planteamiendo del problema}

Para resolver el problema existe un conjunto infinito A1 que tiene los elementos ancestrales a traducir $A1 = \{mother, father, grandmother, greatgrandfather, ...\}$, 
detro de este conjunto se identifico las palabras con las que la gramatica trabaja: father, mother, great y grand.

Con estos datos se analizo la formacion de una traduccion intermedia de tipo:
$mother = mo() o father = fa()$ o,
$grandmother = g(mo())$. 

Al tener las palabras pertenecientes a la gramatica y la conversion de palabra a representacion intermedia, es posible diseñar la gramatica adecuada para este lenguage. La gramatica diseñada seria de este modo:

$$S-> G M | G F  $$
$$G-> R D $$
$$R-> R A $$
$$F-> father$$
$$M-> mother$$
$$D-> grand$$
$$A-> great$$

Al tener las reglas por las cuales la gramatica se forma es posible comenzar de derecha a izquierda a analizar la palabra.

En la segunda parte del problema se pide analizar la respuesta de la primera parte y traducirlo. Para esto la funcion debe hacer uso de una gramatica similar a la usada para los imput's

$$S-> G M | G V  $$
$$G-> R D $$
$$R-> R A $$
$$V-> vatter$$
$$M-> mutter$$
$$D-> gross$$
$$A-> ur$$

Con la gramatica anterior y el resultado de la funcion se logro traducir la frase y encontrar una representacion intermedia.

Finalmente, se recibe una oracion cullas palabras estan dentro del conjunto A3, siendo $A3 = \{The, mother, father, of, Mary, John \}$. Esta oracion pasara traducirce al lenguage A4, donde $A4 = \{Die, ein, mutter, vatter, von, Maria, John \}$

La gramatica del lenguage A3 es:

$$Sentence    -> Init Sentence2$$
$$Sentence2   -> Second Sentence2 | O Nombre $$
$$Nombre      -> Mary | John$$
$$Init        -> The mother | The father $$
$$Second      -> of the mother | of the father$$ 
$$O           -> of$$

La gramatica del lenguage A4 es:
$$Sentence2   -> Second.Sentence2 | G O Nombre $$
$$Nombre      -> Mary | John$$
$$Second      -> ur $$ 
$$O           -> von$$
$$G -> gross$$

Con estas gramaticas creadas solo es necesario realizar un codigo que al leer las frases verifique que sean correctas, mientras se encarga de traducirlas.

\section{Resolucion}

El problema lo dividimos en dos pricipales partes, donde la primera se usa dos clases y en la segunda dos funciones. Se usa la funcion $$string \  middleInterpretation(string \ word)$$
donde recibe como parametro $word$ que es la palabra que pasara a ser una funcion. 

\begin{lstlisting}

// functions.cpp
string middleInterpretation(string word){
	bool controlFoM = false, controlG = false;
	string partialWord;
	string middle;
	for (auto letter = word.rbegin(); letter != word.rend() ; ++letter) {
		partialWord.insert(partialWord.begin(),*letter);
		if (partialWord == "father" and !controlFoM){
			middle.insert(0,"fa()");
			partialWord = "";
			controlFoM = true;
		}else if(partialWord == "mother" and !controlFoM){
			middle.insert(0,"mo()");
			partialWord = "";
			controlFoM = true;
		} else if(partialWord == "grand" and !controlG){
			middle.insert(0,"g(");
			middle.insert(middle.end(),')');
			partialWord = "";
			controlG = true;
		}else if(partialWord == "great" and controlG){
			middle.insert(0,"g(");
			middle.insert(middle.end(),')');
			partialWord = "";
		}else{
			if(partialWord.size() > 6 ){
				cout<<"Error en la frase: "<<partialWord<<endl;
				return "error";
			}
		}
	}
	if(!partialWord.empty()){
		cout<<"Error en la frase: "<<partialWord<<endl;
		return "error";
	}
	return middle;
}

\end{lstlisting}
	
En esta funcion, la variable locales controFoM se encarga de controlar si la frase ya recibio "father" o "mother", ya que esto se debe recibir solo una vez. Por otro lado, controlG se encarga de asegurarse que se reciba solo una vez "grand" y los demas sean "great". Luego, partialWord es la palabra que se forma concatenando cada caracter de derecha a izquierda. Los casos posibles en que la palabra parcial sea distinta a lo esperado se retorna "error". Si todo finalizaba como lo esperaba se devolvia la representacion intermedia.

Para la segunda parte del problema se uso una funcion que recibia la representacion intermedia y luego la traducia a aleman. Para esto se uso la funcion: 
$$string\  finalInterpretation(string\  word) $$
. La funcion recibe el string $word$ que es la representacion intermedia a traducir. 

\begin{lstlisting}
string finalInterpretation(string word){
int count = 0;
bool controlFoM = false, controlG = false;
string partialWord;
string middle;
for (auto letter = word.rbegin(); letter != word.rend() ; ++letter) {
partialWord.insert(partialWord.begin(), *letter);
if (partialWord == ")") {
count += 1;
partialWord = "";
}else if (partialWord == "fa(" and !controlFoM and count>0){
middle.insert(0,"vatter");
partialWord = "";
count -=1;
controlFoM = true;
}else if(partialWord == "mo(" and !controlFoM and count>0){
middle.insert(0,"mutter");
partialWord = "";
controlFoM = true;
count -=1;
} else if( partialWord == "g(" and controlFoM and !controlG and count>0){
middle.insert(0,"gross");
partialWord = "";
controlG = true;
count -=1;
}else if (partialWord == "g(" and controlFoM and controlG and count>0){
middle.insert(0,"ur");
partialWord = "";
count -=1;
}else{
if(partialWord.size() > 3 or count == 0){
cout<<"Error en la frase: "<<partialWord<<endl;
return "error";
}
}
}
return middle;
}
\end{lstlisting}



\section{Concluciones}


\section*{References}

Please number citations consecutively within brackets \cite{b1}. The 
sentence punctuation follows the bracket \cite{b2}. Refer simply to the reference 
number, as in \cite{b3}---do not use ``Ref. \cite{b3}'' or ``reference \cite{b3}'' except at 
the beginning of a sentence: ``Reference \cite{b3} was the first $\ldots$''

Number footnotes separately in superscripts. Place the actual footnote at 
the bottom of the column in which it was cited. Do not put footnotes in the 
abstract or reference list. Use letters for table footnotes.

Unless there are six authors or more give all authors' names; do not use 
``et al.''. Papers that have not been published, even if they have been 
submitted for publication, should be cited as ``unpublished'' \cite{b4}. Papers 
that have been accepted for publication should be cited as ``in press'' \cite{b5}. 
Capitalize only the first word in a paper title, except for proper nouns and 
element symbols.

For papers published in translation journals, please give the English 
citation first, followed by the original foreign-language citation \cite{b6}.

\begin{thebibliography}{00}
\bibitem{b1} G. Eason, B. Noble, and I. N. Sneddon, ``On certain integrals of Lipschitz-Hankel type involving products of Bessel functions,'' Phil. Trans. Roy. Soc. London, vol. A247, pp. 529--551, April 1955.
\bibitem{b2} J. Clerk Maxwell, A Treatise on Electricity and Magnetism, 3rd ed., vol. 2. Oxford: Clarendon, 1892, pp.68--73.
\bibitem{b3} I. S. Jacobs and C. P. Bean, ``Fine particles, thin films and exchange anisotropy,'' in Magnetism, vol. III, G. T. Rado and H. Suhl, Eds. New York: Academic, 1963, pp. 271--350.
\bibitem{b4} K. Elissa, ``Title of paper if known,'' unpublished.
\bibitem{b5} R. Nicole, ``Title of paper with only first word capitalized,'' J. Name Stand. Abbrev., in press.
\bibitem{b6} Y. Yorozu, M. Hirano, K. Oka, and Y. Tagawa, ``Electron spectroscopy studies on magneto-optical media and plastic substrate interface,'' IEEE Transl. J. Magn. Japan, vol. 2, pp. 740--741, August 1987 [Digests 9th Annual Conf. Magnetics Japan, p. 301, 1982].
\bibitem{b7} M. Young, The Technical Writer's Handbook. Mill Valley, CA: University Science, 1989.
\end{thebibliography}
\vspace{12pt}
\color{red}
IEEE conference templates contain guidance text for composing and formatting conference papers. Please ensure that all template text is removed from your conference paper prior to submission to the conference. Failure to remove the template text from your paper may result in your paper not being published.

\end{document}
